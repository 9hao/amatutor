%!TEX encoding = UTF-8 Unicode
\chapter{静态分析}
静态分析是指不将样本中的代码实际运行起来的逆向分析。它的优点包括:
\begin{itemize}
  \item 理论上可以得到样本的全部代码和行为
  \item 不需要真实的运行环境
  \item 分析速度较快
  \item 适合自动化分析
  \item 是检测技术的主要基础
\end{itemize}

但静态分析也不是万能的,主要问题是代码比较复杂时,静态分析需要很长的时间,而也许将样本在模拟器中运行一会儿就能得到更直观的结果。此外,如果样本有使用代码混淆,静态分析的结果将极难读懂。

对Android样本的静态分析包括:
\begin{itemize}
  \item 对Android XML格式的解码
  \item 对DEX文件格式的解析
  \item 对Dalvik指令的“反汇编”(即等价转换为适合于人阅读的形式)
  \item 对Dalvik指令的反编译
  \item 对ELF文件格式的解析
  \item 对ARM指令的反汇编
  \item 对ARM指令的反编译
\end{itemize}

其中,对ARM的静态分析我们将在后续章节单独介绍。
\section{XML解码}
\label{Sec:xml-decode}
\subsection{AXMLPrinter2}
Android中大量使用XML用于,用于项目配置(AndroidManifest.xml)、界面布局、字符串资源等。其中,AndroidManifest.xml中描述了应用程序的基本组件及其参数,这些基本组件可以说是恶意代码分析的“入口点”;此外,其中还包括应用程序运行所申请的额外的安全权限,这也是Android系统安全机制的一个重要组成部分。

但在APK文件中,这些XML文件并不是直接被压缩进去的,而是先被转成了一种特殊的格式(通常称之为AXML格式\index{AXML}),然后再被打包压缩到APK文件。在Android系统中,这些AXML格式文件也不会被还原为XML明文,而是直接进行解析。可惜的是,这种格式的文件,其内容无法直接被阅读。

为什么需要这么做?传统的XML解析有两种模型:SAX(Simple API for XML)和DOM(Document Object Model)。SAX的主要思路是遍历整个XML文件,每次遇到不同的对象(例如一个属性、一个节点开始、一个节点结束等),就触发相应的回调函数对其进行处理。这种解析方法的内存占用很少,适合于只需要遍历XML文件一次或少数几次就能把需要的事情都处理完的应用场景。但在Android中,由于AndroidManifest.xml会被多次使用(比如说,每次调用敏感API时都要检查其权限),所以并不适合使用SAX。DOM的主要思想则是将XML视为一棵树,在内存中建立并长期维护这颗树,而所有对XML的解析都变成对这颗树的访问。它的优点是可以每次访问都很快,甚至可以并行地方法(存在一定的读写互斥问题,这是另一个很有意思的话题)。但它的缺点也很明显,即需要长期占用较大的内存,对手机设备来说,内存是很重要的资源,这种消耗是应该要避免的。

实际上还有一种解析模型,叫PULL\footnote{\url{http://www.xmlpull.org}},它是SAX和DOM的折中方案。它很类似于SAX的事件触发机制,但这种触发并不调用事先注册的回调函数,而是由解析者自己触发事件并根据实际情况做自己的事情。由于这种主动性,即解析者可以自己决定哪些地方是感兴趣的、哪些地方不是,所以可以非常轻量地跳转到自己需要的那一部分数据做解析。

已经说得比较远了。之所以介绍PULL,是因为Android不但使用了PULL,还为PULL模型专门设计了AXML格式,这种格式的PULL解析非常容易实现,解析速度很快。此外,AXML还将所有字符串使用UTF-16LE编码、所有字段4字节内存对齐(Dalvik虚拟机和ARM处理器在访问4字节对齐数据时都会更快)、对重复字符串资源做成单实例等优化、根据需要解析字符串等。通过这些优化,进一步提高了Android系统解析XML文件的效率。

AXMLPrinter是开源项目android4me\footnote{\url{http://code.google.com/p/android4me/}}的一个产品,可以将从XML到AXML的编码过程逆向执行,即将AXML文件还原为可读的XML形式。它的使用非常简单:
\begin{lstlisting}[language=bash, numbers=none]
java -jar AXMLPrinter2.jar encoded.xml >decoded.xml
\end{lstlisting}

AXML格式解析的C语言实现可以使用我写的AndTools\footnote{\url{https://www.github.com/claudxiao/AndTools}},比较适合用于实现自动化工具,但稳定性还不是很好。除了这份代码,在AXMLPrinter2中有Java版的解析代码,在后面要介绍的androguard中有Python版的解析代码。

\section{Dalvik反汇编}
\subsection{smali}
apk文件中的“代码”实际上都存储在了classes.dex文件之中。这个文件类似于Java的类文件,是Dalvik虚拟机上的二进制可执行文件。smali和下一节介绍的dedexer是两款逆向.dex文件的工具,输出可读性较好的Dalvik指令文件。

smali\footnote{\url{http://code.google.com/p/smali/}}实际上由两个工具smali.jar和baksmali.jar组成。它使用基于jasmin\footnote{\url{http://jasmin.sourceforge.net}}\index{jasmin}的汇编语法。其中,baksmali.jar用于反汇编:
\begin{lstlisting}[language=bash, numbers=none]
java -jar baksmali.jar classes.dex -o dir
\end{lstlisting}
上述命令会在dir目录下输出classes.dex的反汇编结果,其中文件扩展名为.smali,每个Java类一个文件,并以Java package名来组织路径。例如,com.example.Hello类将生成com/example/Hello.smali文件。

smali.jar则用于将.smali文件再次编码为.dex文件:
\begin{lstlisting}[language=bash, numbers=none]
java -jar baksmali.jar dir -o classes_new.dex
\end{lstlisting}

因为精确地汇编与反汇编,smali成为Android病毒编写和分析中最常用的工具之一。
\subsection{IDA}
从IDA 6.1开始,支持对DEX文件格式的解析,以及对Dalvik指令集的反汇编。使用IDA直接打开apk文件或classes.dex文件,IDA会自动识别DEX格式。

使用IDA分析DEX文件的主要优点是会自动识别函数调用,并建立交叉索引关系。这一点在DEX文件被ProGuard混淆后非常有用。因为代码混淆就是通过移除有效的名称信息,降低逆向分析的可理解性,并提高人工查找交叉引用关系和搜索名字的难度。
\subsection{dedexer}
dedexer\footnote{\url{http://sourceforge.net/projects/dedexer/}}是另一个.dex文件反汇编工具,用法是:
\begin{lstlisting}[language=bash, numbers=none]
java -jar ddx.jar -d dir classes.dex
\end{lstlisting}
如果使用\lstinline!-o!参数,还将在dir目录下输出名为dex.log的.dex解析日志。
\section{Dalvik反编译}
将Java源码编译为Dalvik指令分为两步:先使用\lstinline!javac!将源码编译为.class文件,再使用\lstinline!dx!转换为DEX文件。反之,将DEX文件反编译为Java源码通常也是两步:将DEX再次转换为.class或.jar文件,主要是指令和文件格式的转换;然而将.class或.jar文件用Java反编译器反编译成Java源码。

\subsection{dex2jar}
dex2jar\footnote{\url{http://code.google.com/p/dex2jar/}}将classes.dex逆向为Java的类文件(.class文件),并打包为.jar格式。jar格式本质上就是按package路径的类文件的ZIP压缩。

dex2jar并不直接使用classes.dex,它自带了解包功能,命令为:
\begin{lstlisting}[language=bash, numbers=none]
dex2jar Example.apk
\end{lstlisting}
即直接作用于apk文件,在当前目录生成名为Example.apk.dex2jar.jar的文件。

应当注意的是,dex2jar工具的逆向并不完全精确。目前已知在转换switch、try-exception等函数结构时,将会出错。
\subsection{jd-gui}
jd-gui\footnote{\url{http://java.decompiler.free.fr/?q=jdgui}}是一个图形界面的Jar文件反编译器。我们通常用jd-gui来反编译dex2jar产生的.jar文件,得到Java代码。

dex2jar对部分函数结构的逆向错误将直接在jd-gui的反编译结果中显示出来。此外,部分特殊优化的类方法(函数),jd-gui的反编译将失败。

\subsection{ded}
ded\footnote{\url{http://siis.cse.psu.edu/ded/}}\index{ded}是宾西法尼亚大学研究人员开发的软件,可以将DEX文件转为Java的.class文件。目前只能在Linux和Mac OS X下使用。

ded使用soot\footnote{\url{http://www.sable.mcgill.ca/soot/}}\index{soot}对.class文件做优化,优化后反编译得到的Java源文件效果较好,在一些细节上优于dex2jar和jd-gui的组合,尤其是对一些循环结构和异常处理代码的反编译上。但soot的运行速度极慢、占用内存较多。

\subsection{其他}
还有undx\footnote{\url{http://sourceforge.net/projects/undx/}}\index{undx}和dex-decompiler\footnote{\url{http://code.google.com/p/dex-decompiler/}}\index{dex-decompiler}这两个反编译器,但效果如何,还没有试用过。

除了jd-gui和soot,常用的Java反编译工具还有jad\footnote{\url{http://www.varaneckas.com/jad}}\index{jad}和dava\footnote{\url{http://www.sable.mcgill.ca/dava/}}\index{dava}。

\section{ARM分析}
\subsection{IDA}
除了classes.dex,在恶意代码中还可能存在另一种可执行文件:ARM架构下ELF格式可执行文件。事实上,恶意代码经常出于以下三个目的使用这类文件:
\begin{itemize}
	\item 将部分攻击行为用NDK开发,增加分析难度
	\item 释放(Linux下可执行的)提权工具
	\item 某些攻击行为必须用到底层的可执行文件,例如系统模块替换
\end{itemize}
因此,我们需要逆向分析这类文件。

IDA Pro\footnote{\url{http://www.hex-rays.com/idapro/}}是Hex\-Rays公司开发的反汇编器。最新的6.1版需要购买,但5.0版可以免费使用。

在IDA的菜单中,依次选择File$\rightarrow$New$\rightarrow$Unix$\rightarrow$ELF/COFF Dynamic Library$\rightarrow$OK,选择动态链接库文件即可开始分析。IDA会自动判断代码的体系结构,并选择相应的反汇编模块。
\subsection{toolchain}

\section{数字签名}
\subsection{signapk}
signapk是Android源码中为APK格式文件签名的程序,其源码位于build/tools/signapk/目录,编译后的结果位于的out/host/linux-x86/framework/目录。用法是:
\begin{lstlisting}[language=bash, numbers=none]
java -jar signapk.jar publickey.x509[.pem] privatekey.pk8 input.apk output.apk
\end{lstlisting}
APK文件经过signapk签名,再使用\lstinline!zipalign!对齐,就可以安装到手机中了。

\subsection{openssl}
一般情况下,APK文件的签名位于其压缩包的META-INF/CERT.RSA文件中。可以使用openssl\index{openssl}查看其签名信息:
\begin{lstlisting}[language=bash, numbers=none]
 $ openssl pkcs7 -in CERT.RSA -inform DER -print_certs
\end{lstlisting}

\section{综合工具}
\subsection{apktool}
apktool\footnote{\url{http://code.google.com/p/android-apktool/}}实际上是一系列第三方工具的集合。它的功能包括:apk文件解包和打包、XML文件编码与反编码、resources.arsc的解包和打包、smali汇编与反汇编、samli调试等。

其中,最常用的功能是解包和打包。解包命令为:
\begin{lstlisting}[language=bash, numbers=none]
apktool d Example.apk dir
\end{lstlisting}
会将Example.apk文件解包到dir目录,并自动完成其中XML文件、资源文件、.dex文件的后续处理。如果省略dir参数,则自动解包到.apk文件名的目录Example。

打包命令为:
\begin{lstlisting}[language=bash, numbers=none]
apktool b dir New.apk
\end{lstlisting}

apktool提供了对smali文件的实时调试功能。[TODO:待补充]
\subsection{androguard}
\subsection{apkinspector}
\subsection{dexid}

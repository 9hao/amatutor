%!TEX encoding = UTF-8 Unicode
\chapter{静态分析}
\section{apk文件格式}
Android系统使用的安装包文件后缀名为.apk。实际上它是标准的ZIP格式文件,可以使用7-zip等压缩软件、zlib等函数库进行处理
\footnote{ZIP格式的详细介绍,参考\url{http://en.wikipedia.org/wiki/ZIP\_(file\_format)}。}。

一个.apk文件解压缩后,通常包括如下文件或文件夹:
\begin{itemize}
	\item[-] AndroidManifest.xml:项目配置文件,但不是明文的XML文件,直接打开无法阅读
	\item[-] classes.dex\index{classes.dex}:所有的Java类文件被打包并转换为这个文件,它是Dalvik虚拟机上的可执行二进制文件,Android系统在运行时将其动态优化为.dey文件并投入一个Dalvik虚拟机实例中运行。
	\item[-] resources.arsc:资源文件打包而成,字符串值(源码中res/value/Strings.xml)就在其中
	\item[-] res/:资源文件夹,其中包括布局文件夹layout、图标图片文件夹drawable、原始文件文件夹raw(该目录下任何文件都会原封不动地存储到设备上)等
	\item[-] META-INF/:数字签名文件夹
	\item[-] lib/:包含NDK开发中使用的动态链接库文件(ARM架构下ELF格式.so文件),有的软件没有这个文件夹
	\item[-] assets/:原始文件文件夹,其中的文件不会被压缩,也不能想res/目录下的资源文件一样通过R类引用,有的软件没有这个文件夹
\end{itemize}
其中,数字签名文件夹META-INF/包含下列文件:
\begin{itemize}
	\item[-] CERT.SF:生成每个文件相对的密钥
	\item[-] MANIFEST.MF:数字签名信息
	\item[-] xxx.SF:这是 JAR 文件的签名文件,xxx标识了签名者
	\item[-] xxx.DSA:对输出文件的签名和公钥
\end{itemize}

\section{XML解码}
\subsection{AXMLPrinter2}
Android使用了大量XML文件用于项目配置、布局、文本资源等。XML具有较高的可读性,但格式解析的开销较大,适合开发人员编写,但不适合“寸土寸金”的手机解析。因此,Android在编译时将XML文件编码为另一种格式,并使用更快的算法解析格式。编码后的XML文件无法直接阅读。

AXMLPrinter是开源项目android4me\footnote{\href{http://code.google.com/p/android4me/}{http://code.google.com/p/android4me/}}的一个产品,可以将上述编码过程逆向执行,即将编码后的XML文件还原为可读形式。它的使用非常简单:
\begin{lstlisting}[language=bash, numbers=none]
java -jar AXMLPrinter2.jar encoded.xml >decoded.xml
\end{lstlisting}
\section{DEX反汇编}
\subsection{smali}
apk文件中的“代码”实际上都存储在了classes.dex文件之中。这个文件类似于Java的类文件,是Dalvik虚拟机上的二进制可执行文件。smali和下一节介绍的dedexer是两款逆向.dex文件的工具,输出可读性较好的Dalvik指令文件。

smali\footnote{\url{http://code.google.com/p/smali/}}实际上由两个工具smali.jar和baksmali.jar组成。其中,baksmali.jar用于反汇编:
\begin{lstlisting}[language=bash, numbers=none]
java -jar baksmali.jar classes.dex -o dir
\end{lstlisting}
上述命令会在dir目录下输出classes.dex的反汇编结果,其中文件扩展名为.smali,每个Java类一个文件,并以Java package名来组织路径。例如,com.example.Hello类将生成com/example/Hello.smali文件。

smali.jar则用于将.smali文件再次编码为.dex文件:
\begin{lstlisting}[language=bash, numbers=none]
java -jar baksmali.jar dir -o classes_new.dex
\end{lstlisting}

因为精确地汇编与反汇编,smali成为Android病毒编写和分析中最常用的工具之一。
\subsection{IDA}
\subsection{dedexer}
dedexer\footnote{\url{http://sourceforge.net/projects/dedexer/}}是另一个.dex文件反汇编工具,用法是:
\begin{lstlisting}[language=bash, numbers=none]
java -jar ddx.jar -d dir classes.dex
\end{lstlisting}
如果使用\lstinline!-o!参数,还将在dir目录下输出名为dex.log的.dex解析日志。
\section{DEX反编译}
\subsection{dex2jar}
dex2jar\footnote{\url{http://code.google.com/p/dex2jar/}}将classes.dex逆向为Java的类文件(.class文件),并打包为.jar格式。jar格式本质上就是按package路径的类文件的ZIP压缩。

dex2jar并不直接使用classes.dex,它自带了解包功能,命令为:
\begin{lstlisting}[language=bash, numbers=none]
dex2jar Example.apk
\end{lstlisting}
即直接作用于apk文件,在当前目录生成名为Example.apk.dex2jar.jar的文件。

应当注意的是,dex2jar工具的逆向并不完全精确。目前已知在转换switch、try-exception等函数结构时,将会出错。
\subsection{jd-gui}
jd-gui\footnote{\url{http://java.decompiler.free.fr/?q=jdgui}}是一个图形界面的Jar文件反编译器。我们通常用jd-gui来反编译dex2jar产生的.jar文件,得到Java代码。

dex2jar对部分函数结构的逆向错误将直接在jd-gui的反编译结果中显示出来。此外,部分特殊优化的类方法(函数),jd-gui的反编译将失败。
\subsection{ded}
\subsection{soot}
\subsection{undx}

\section{ARM分析}
\subsection{IDA}
除了classes.dex,在恶意代码中还可能存在另一种可执行文件:ARM架构下ELF格式可执行文件。事实上,恶意代码经常出于以下三个目的使用这类文件:
\begin{itemize}
	\item 将部分攻击行为用NDK开发,增加分析难度
	\item 释放(Linux下可执行的)提权工具
	\item 某些攻击行为必须用到底层的可执行文件,例如系统模块替换
\end{itemize}
因此,我们需要逆向分析这类文件。

IDA Pro
\index{IDA Pro}
\footnote{\url{http://www.hex-rays.com/idapro/}}
是Hex\-Rays公司开发的反汇编器。最新的6.1版需要购买,但5.0版可以免费使用。具体使用方法可以参考\cite{IDA_PRO_BOOK}。

在IDA的菜单中,依次选择File$\rightarrow$New$\rightarrow$Unix$\rightarrow$ELF/COFF Dynamic Library$\rightarrow$OK,选择动态链接库文件即可开始分析。IDA会自动判断代码的体系结构,并选择相应的反汇编模块。
\subsection{toolchain}

\section{数字签名}
\subsection{signapk}
\subsection{openssl}

\section{综合工具}
\subsection{apktool}
apktool\footnote{\url{http://code.google.com/p/android-apktool/}}实际上是一系列第三方工具的集合。它的功能包括:apk文件解包和打包、XML文件编码与反编码、resources.arsc的解包和打包、smali汇编与反汇编、samli调试等。

其中,最常用的功能是解包和打包。解包命令为:
\begin{lstlisting}[language=bash, numbers=none]
apktool d Example.apk dir
\end{lstlisting}
会将Example.apk文件解包到dir目录,并自动完成其中XML文件、资源文件、.dex文件的后续处理。如果省略dir参数,则自动解包到.apk文件名的目录Example。

打包命令为:
\begin{lstlisting}[language=bash, numbers=none]
apktool b dir New.apk
\end{lstlisting}

apktool提供了对smali文件的实时调试功能。[TODO:待补充]
\subsection{androguard}
\subsection{apkinspector}
\subsection{dexid}

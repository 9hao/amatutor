%!TEX encoding = UTF-8 Unicode
\chapter{动态分析}

\section{运行环境}
\subsection{emulator}
Android SDK中的模拟器emulator是样本动态运行的最佳环境。它的缺点和不足如下:
\begin{itemize}
\item 虽然可以模拟短信和通话,但并没有真正和移动通信网络相连
\item 对依赖于ROM环境、依赖于特定手机型号的样本很难发挥作用
\end{itemize}
\subsection{手机}
手机是样本运行的真实环境,并且可以与运营商网络直接相连。但硬件成本高。
\section{设备管理}
\subsection{adb}
adb是连接PC与设备(模拟器或手机)的主要工具。它的功能包括:
\begin{itemize}
\item 交互式shell
\item logcat
\item 安装、卸载软件
\item 传输调试信息
\item 传输文件
\end{itemize}
\subsection{ddms}
SDK中的ddms提供了一个细粒度的可视化调试环境。
\section{网络分析}
\subsection{tcpdump}
为了捕获样本在运行期间产生的网络数据,需要对其抓包。主要工具是tcpdump。针对不同的场景,可以在三个不同的位置使用这一工具。
\subsubsection{模拟器}
模拟器emulator有一个没有公开的参数-tcpdump,在启动模拟器时,通过该参数指定一个本地文件路径,可以将模拟器运行期间产生的所有网络数据捕获到指定的pcap文件。
\subsubsection{Android系统}
有移植到Android底层的原生tcpdump工具,但其运行需要root权限。这一工具可以在模拟器中的系统里运行,但更多时候用于真实手机系统的抓包。
\subsubsection{PC网络}
当真实手机使用Wi-Fi通信,可以在无线网络或者其后端的物理网络抓包。
\subsection{wireshark}
对网络数据的分析一般使用wireshark这一传统的网络分析工具。最常用的功能是其filter。

\section{行为模拟}
\subsection{am}

\subsection{DNS}

\subsection{telnet}
通过telnet可以与模拟器中的Android系统收发短信、拨打接听电话。
\subsection{openBTS}
对真实手机中的Android系统,可以通过架设一个OpenBTS系统来获取其发送短信、拨打电话的行为,并模拟任意的外部号码向其发送短信、拨打电话。

\section{行为跟踪}
\subsection{tanitdroid}
\subsection{droidbox}

\section{调试}
\subsection{IDA}
\subsection{AndBug}
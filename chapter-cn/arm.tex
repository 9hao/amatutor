%!TEX encoding = UTF-8 Unicode
\chapter{ARM漏洞利用}
\section{学习的目的}
目前绝大部分的智能移动终端使用ARM处理器。包括使用Android、Apple iOS、Symbian、Windows Mobile/Windows Phone、RIM OS等操作系统的手机和平板电脑。

此外,ARM还被用于大量的嵌入式设备,在PC所不能及之处发挥着重要的作用。

PC也可以使用ARM。几年前,Debian就发布了ARM版,2011年Ubuntu也开始支持ARM。Windows 8也将被微软移植到ARM上。

ARM还将向服务器市场发展,目前还出现了基于ARM和GPU的超级计算机。

可以看出,ARM设备无论是数量还是覆盖面都不逊色于PC,但ARM的安全问题,尤其是其上软件漏洞的问题,直到最近几年才开始引起人们的注意。
\section{环境和工具}
\subsection{容易获得的ARM系统}
如果需要自己编写的ARM代码实际运行起来,或者调试这些代码,就需要一个能运行ARM代码的系统。ARM的开发板并不贵,但还有更简单的方法获得这样一个系统。
\subsubsection{Android模拟器或手机}
Android的底层是Linux内核,它运行于ARM之上。使用Android学习ARM漏洞利用的优点是:
\begin{itemize}
\item 如果就是要做Android安全,这自然是最佳的选择
\item 有完善的开发工具和系统源码
\item Android模拟器的硬件(边际)成本为0
\end{itemize}
但缺点也不少:
\begin{itemize}
\item 缺乏丰富的本地软件
\item 不使用glibc库,而是自行开发的bionic,因此一些利用技巧在细节上会和Linux(on ARM)有差异
\item 目标文件的编译,或者完全静态链接,或者用NDK开发,或者与Android源码一起编译,均显繁琐
\end{itemize}

无论如何,这种零成本的环境,还是强烈推荐常备着。
\subsubsection{Debian虚拟机}
Debian很早就开始了对ARM处理器的移植,目前几乎所有的软件都有了ARM版本,因此Linux用户常用的工具(例如objdump、gdb等)都可以直接在本地运行。

另一方面,qemu模拟器已经可以完整地模拟ARM处理器及其硬件平台,可以用它在x86的PC上运行ARM版的Debian。

可以直接从Debian的官网下载到其ARM安装包,但建议使用下列方案:
\hrefurl{http://www.aurel32.net/info/debian\_arm\_qemu.php}

更省事的方案是直接下载已经安装好的qemu镜像,地址是:
\hrefurl{http://people.debian.org/~aurel32/qemu/arm/}
\subsubsection{Toshiba AC100上网本}
Toshiba AC100是目前唯一一款拥有标准全键盘的ARM笔记本。它最初是为Android设计,但爱好者已经将多个Ubuntu移植到该机器上。最终,Ubuntu官方从11.10版本开始专门为这一机器提供预编译版本和软件源。

该机器在国内的型号为AC100-01B,已经不在出新货,可以从爱好者手中购买到二手的,2011年年底的行情为750元左右。

这个笔记本是目前唯一一种可以最轻松获得的真实ARM机器。拥有前面Debian虚拟机的一切优点。
\subsubsection{Nokia N900手机}
另一个hack利器是Nokia N900手机,这款老手机配置较高,采用已经被抛弃的Maemo系统,该系统是基于Linux的手机操作系统,开放性较高。N900的键盘操作性不强,系统也适合于有hack精神的人玩。
\subsection{交叉编译工具}
进行嵌入式开发时,编译工具运行于一个平台,但生成另一个平台的指令,这种编译过程称之为交叉编译,所使用的编译工具又称之为工具链。

在x86上编译ARM代码最常用的工具链由CodeSourcery公司免费提供,有Windows版本和Linux版本,其下载地址是:\hrefurl{http://www.mentor.com/embedded-software/sourcery-tools/sourcery-codebench/editions/lite-edition}

此外,在基于Debian的系统(例如Ubuntu)中,也可以通过apt-get install gcc-arm-linux-gnueabi直接获得一个交叉编译gcc。

Android的NDK中实际上也包含了完整的工具链,包括编译器、调试器、binutils等。它有Windows和Linux版本,下载地址是:
\hrefurl{http://developer.android.com/sdk/ndk/index.html}
然而NDK也有不足,即只能为Android编译代码,也只能使用其提供的有限的库接口。对后面一个问题,如果需要编译使用了Andrid系统中较底层API的ARM程序,建议使用agcc工具,具体可以看我写的文章:
\hrefurl{http://blog.claudxiao.net/2011/10/android\_agcc/}
\subsection{反汇编和反编译工具}
工具链包含了很多的工具,如果静态反汇编,可以使用objdump;如果动态运行起来,在gdb中也可以使用disasm反汇编。

最ARM反汇编支持最好的工具是IDA Pro,但它是商业软件,自由开源的反编译工具可以考虑radare:
\hrefurl{http://www.radare.org/y/}
或者smiasm:
\hrefurl{http://code.google.com/p/smiasm/}
但radare和smiasm对Thumb、Thumb-2指令集的支持都不强。

事实上,ARM的指令编码并不复杂,自己写一个ARM反汇编器的工作量远小于想象。

目前ARM的反编译工具只有随IDA Pro 6.2一起发布的Hex-Rays ARM Decompiler,是一款昂贵的收费软件。
\subsection{远程调试}
gdb远程调试

IDA远程调试请参考:
\hrefurl{http://bbs.pediy.com/showthread.php?t=141739}

\section{ARM体系结构}
\subsection{ARM指令集}
ARM同时是三个不同事物的名字:
\begin{itemize}
\item 一个公司
\item 一种体系结构
\item 一系列CPU产品
\end{itemize}
这一章主要是指第二种含义。

ARM是32位RISC结构,在ARMv5以后,基本采用Harvard结构,与传统的冯诺依曼结构不同的是,数据和代码被最大程度隔离了。既便于数据段保护的x86也有本质区别,ARM的代码和数据采用不同的总线传输(并因此获得并行而提速)。其数据段自然不能被执行。

\subsubsection{寄存器}
对程序员可见的寄存器主要是r0到r15共16个,不同的寄存器有不同的用途,将在下一节说明。

此外,有程序状态寄存器PSR,其中包括算数逻辑运算标志、执行状态位、当前中断号等。

\subsubsection{数据操作指令}
基本形式是:
\begin{lstlisting}[numbers=none]
<opcode>{<cond>}{S} <Rd>, <Rn>{, <operand2>}
\end{lstlisting}
其中:
\begin{itemize}
\item[opcode] 比如MOV、ADD,与x86类似,具体可以查ARM手册
\item[cond] 可选的条件码。条件执行是ARM的特色之一,几乎所有的ARM指令都可以加上条件码,实现条件执行。例如,EQ、LE等
\item[S] 可选的S位,如果该条指令会更改PSR,则应置上S位
\item[operand] 操作数可以是寄存器,也可以是一个右值(后面会说明),每条指令有多少操作数,分别是什么含义,应查手册
\end{itemize}

ARM中的右值操作数可以是一个寄存器、一个立即数,或者一个寄存器的移位。例如,右值R1, LSL \#2由R2的值逻辑左移2位得到,右值R1, ASR R3由R1的值算术右移R3位得到。

\subsubsection{分支跳转指令}
\begin{itemize}
\item[B] 直接跳转到一个地址,唯一的参数是12位立即数,与x86中的jmp功能一样
\item[BL] 将下一条指令地址(即返回地址)赋给LR寄存器,然后跳转到参数指定的地址,参数是一个12位立即数。BL与x86中的call功能一样
\item[BX] 参数是一个寄存器,跳转到该寄存器指向的地址
\end{itemize}
所有这些分支跳转指令都可以使用条件码,成为相应的条件跳转指令。

\subsubsection{内存访问指令1}
ARM使用简单的访存模型,所有内存读写都通过LDR和STR两条指令完成,而且只能在寄存器与内存之间读写,不能直接从内存读写到内存。LDR和STR都有一个可选的后缀B,不选时按字(4字节)读写,选择B时按字节读写。其第一个参数是寄存器,第二个参数是内存地址。

\subsubsection{寻址方式}
第一类寻址方式:
\begin{itemize}
\item 寄存器加上立即数偏移:[reg, \#$\pm$imm12]
\item 寄存器加上寄存器偏移:[reg, $\pm$reg]
\item 寄存器a加上移位后的寄存器b偏移:[rega, $\pm$regb, shift]
\end{itemize}
这些地址符号后面可以选择一个叹号:!。如果加上,表明先根据寻址规则修改寄存器,然后根据寄存器中的值访问内存;如果不加叹号,表示直接根据寻址规则访问内存。

第二类寻址方式则是先根据寄存器中的值访问内存,然后按照相应的规则更新寄存器:
\begin{itemize}
\item 访存后,寄存器加上立即数:[reg], \#$\pm$imm12
\item 访存后,寄存器加上寄存器:[reg], $\pm$reg
\item 访存后,寄存器a加上移位后的寄存器b:[rega], $\pm$regb, shift
\end{itemize}

\subsubsection{内存访问指令2}
还可以一次性读写多个字:
\begin{lstlisting}[numbers=none]
LDMcdum reg!, mreg
STMcdum reg!, mreg
\end{lstlisting}
其中:
\begin{itemize}
\item[cd] 可选的条件
\item[um] 访问模式,分别为IA读写后增加寄存器值、DA读写后减少寄存器值、IB读写前增加寄存器值、DB读写后增加寄存器值
\item[!] 表示会修改寄存器,修改方法参考um
\item[mreg] 支持一次指令多个寄存器,例如{R0-R3, R7, R10}
\end{itemize}

\subsection{ATPCS}
\subsubsection{名词解释}
例程(routine)、子例程(subroutine):对于一段可以调用、可以返回、保证栈平衡的代码片段,例程指调用者,子例程指被调用者。

过程(procedure):不返回结果值的例程。

函数(function):返回结果值的例程。

变量大小、内存对齐、字节序、复合类型等略。
\subsubsection{寄存器的用途}

ARM中有16个通用寄存器r0 – r15。其中:
\begin{itemize}
\item[r0 – r3] 用于传递参数、返回函数结果,因此又名a0 – a3。在例程内部也被用于保存临时结果

\item[r4 – r11] 用于保存例程内的局部变量值,又名v0 – v8。其中,r9(v6)是一个平台相关的值,不同的ARM平台必须为这个寄存器赋予特殊的含义

\item[r12 – r15] 有专门的用途,后面介绍。常用别名:IP、SP、LR、PC
\end{itemize}
\subsubsection{栈结构}

r13(SP)是栈指针。ARM中使用向下的满栈(full-descending),即SP始终指向最后一个已进入栈的数据,每次压栈时,SP自减需要的内存大小,然后将值存到SP新的位置。

也就是说,对栈的push相当于STMDB,对栈的pop相当于LDMIA,但一般不用这个后缀,而用STMFD和LDMFD。

SP的值应在栈的有效区间内,且mod 4 = 0。栈上有例程的调用帧结构。不要轻易地修改这个值。

\subsubsection{子例程调用}

ARM和Thumb指令集均有一个BL指令,它的操作是:将BL指令顺序下一条指令的地址(即返回地址)送入链接寄存器LR(r14),然后将调用的目的地址(即子例程地址)送入寄存器PC(r15)。

如果在Thumb中调用BL,则LR的第0位被设为1,否则被设为0。(因为指令地址要4字节对齐,因此LR的第0和1位都没有用。)

子例程返回很简单,将LR的值送入PC即可。

任何模拟上述过程的指令序列也会起到BL的效果,例如:mov LR, PC; BX r4。注意,任何时候读PC,得到的值是当前指令地址+8(why?请自己google)。

\subsubsection{结果返回}
\begin{itemize}
\item 不超过4字节的结果,一律用r0返回

\item 超过4字节的基本类型,继续用r1, r2, r3

\item 超过4字节的复合类型,或动态大小的结果,存储在内存中,并将其地址作为调用参数之一传入
\end{itemize}
\subsubsection{参数传递}

使用r0 – r3传递参数,如果不够,使用栈。

语言中的数据类型会按照相关标准转化为机器数据类型。
\section{ARM漏洞利用的特点}
\section{一个示例}
\section{Ret2ZP攻击}
\section{setuid漏洞分析}
\section{zergRush漏洞分析}
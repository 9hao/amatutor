%!TEX encoding = UTF-8 Unicode
\chapter{ARM漏洞利用}
\section{学习的目的}
目前,绝大部分智能移动终端使用ARM处理器,包括基于Android、iOS、Symbian、Windows Mobile、Windows Phone等操作系统的手机和平板电脑。ARM还被大量的嵌入式设备使用。

使用ARM处理器的PC也已经出现。几年前,Debian就发布了ARM移植版;2011年,Ubuntu开始支持ARM;2012年将正式发布的Windows 8也已经被微软移植到ARM上。预计在2012年,使用ARM处理器的笔记本电脑将开始流行。

ARM还开始向服务器市场发展。目前,Nvidia和巴塞罗那超级计算中心还推出了基于ARM和CUDA计算的超级计算机,Nvidia也将在2012年初向普通开发者推出基于ARM的CUDA开发工具。

可以看出,ARM设备无论是数量上还是覆盖面都不逊色于PC。但ARM的安全问题,尤其是其上软件漏洞的问题,直到近几年才引起人们的注意。
\section{环境和工具}
\subsection{ARM设备和系统}
\subsubsection{开发板}
获得一个ARM设备的常规方法是购买或者定制ARM开发板。如果不打算移植Android等系统,可以选择mini2440,否则可以选择mini6410等型号。

需要自己为开发板交叉编译操作系统,并烧录进去。这个过程会比较好玩。但如果觉得乏味,或者时间不够,可以使用Ångström发行版\footnote{\href{http://www.angstrom-distribution.org/}{http://www.angstrom-distribution.org/}}。

开发板的优点是,它是一个真实的ARM设备和运行环境;缺点是操作稍显繁琐。除此以外,还有其他方法可以获得ARM系统。

\subsubsection{Android模拟器或手机}
Android手机本身就是一个ARM设备,而Android底层的Linux内核就是运行于ARM上的系统。因此,Android手机基本是可以充当ARM开发和调试环境的。此外,Android SDK中的模拟器也是基于qemu模拟了ARM设备。从漏洞分析和漏洞利用的角度来看,使用Android手机或模拟器进行学习的优点包括:

\begin{itemize}
\item 如果要分析Android的漏洞,这自然是最佳的选择
\item 有完善的开发工具链和系统源码
\item 模拟器的硬件(边际)成本为0
\end{itemize}

但缺点也不少:

\begin{itemize}
\item 在Android中的Linux上,没有太多的本地应用程序,例如gdb等调试工具(只有gdbserver)
\item 不使用glibc库,而是自行开发的bionic,因此一些利用技巧在细节上会和Linux(on ARM)有差异
\item 目标文件的编译,或者完全静态链接,或者用NDK开发,或者与Android源码一起编译,均显繁琐
\end{itemize}

无论如何,模拟器完全没有成本,建议在PC中常备以便使用。

\subsubsection{Debian虚拟机}
早在2000年,Debian就开始了对ARM处理器的移植工程,目前其仓库中几乎所有的软件都有了ARM版本,包括objdump、gdb等常用开发和漏洞分析工具。

另一方面,qemu模拟器已经可以完美地模拟ARM处理器及其硬件平台,可以用它在x86的PC上运行ARM版的Debian。

可以直接从Debian的官网下载到其ARM安装包,然后参考aurel提供的安装指南\footnote{\href{http://www.aurel32.net/info/debian\_arm\_qemu.php}{http://www.aurel32.net/info/debian\_arm\_qemu.php}}。

更省事的方案是直接下载已经安装好的qemu镜像,也是由aurel提供。包括arm版\footnote{\href{http://people.debian.org/~aurel32/qemu/arm/}{http://people.debian.org/$\sim$aurel32/qemu/arm/}}
和armel版\footnote{\href{http://people.debian.org/~aurel32/qemu/armel/}{http://people.debian.org/$\sim$aurel32/qemu/armel/}}
,建议使用后者。

\subsubsection{Toshiba AC100}
Toshiba AC100是目前唯一一款拥有标准全键盘的ARM笔记本。它最初是为Android设计,但hackers已经将多个Ubuntu版本移植到该机器上。最终,Canonical公司从Ubuntu 11.10开始专门为这一机器提供预编译版本和软件源,作为其向ARM平台扩张的第一步。

该机器在国内的型号为AC100-01B,已经不再生产,但可以购买到二手的。

这个笔记本是目前唯一一种可以最轻松获得的真实ARM机器,并拥有前面提到的Debian虚拟机的一切优点。

此外,在前面所述的方案中,除了4.0以上Android,其Linux Kernel均未开启ASLR(地址空间布局随机化)特性,但在Ubuntu for AC100中,开启了ASLR。这固然为简单的ret2libc等攻击带来了困难,但另一方面,也成为几乎是唯一的分析ASLR下漏洞的环境。考虑到Android 4.0开始去掉prelink优化、全面开启ASLR特性\cite{aslr_android},这种研究的价值是显然的。

\subsubsection{Nokia N900手机}
另一个hack利器是Nokia N900手机,这款老手机采用已经被抛弃的Maemo系统,该系统是基于Linux的手机操作系统,开放性较高。N900的键盘操作性不强,系统也适合于有hack精神的人玩。
\subsection{交叉编译工具}
进行嵌入式开发时,编译工具运行于一个平台,但生成另一个平台的指令,这种编译过程称之为交叉编译,所使用的编译工具又称之为工具链。

在x86上编译ARM代码最常用的工具链由CodeSourcery公司免费提供,有Windows版本和Linux版本\footnote{\href{http://www.mentor.com/embedded-software/sourcery-tools/sourcery-codebench/editions/lite-edition}{http://www.mentor.com/embedded-software/sourcery-tools/sourcery-codebench/editions/lite-edition}}。

此外,在基于Debian的系统(例如Ubuntu)中,也可以通过\lstinline!apt-get install gcc-arm-linux-gnueabi!直接获得一个交叉编译gcc。

Android的NDK中实际上也包含了完整的工具链,包括编译器、调试器、binutils等。它同样有Windows和Linux版本\footnote{\href{http://developer.android.com/sdk/ndk/index.html}{http://developer.android.com/sdk/ndk/index.html}}。

然而NDK也有不足,即只能为Android编译代码,也只能使用其提供的有限的库接口。对后面一个问题,如果需要编译使用了Andrid系统中较底层API的ARM程序,建议使用agcc工具,具体可以看我写的文章\footnote{\href{http://blog.claudxiao.net/2011/10/android_agcc/}{http://blog.claudxiao.net/2011/10/android\_agcc/}}。
\subsection{反汇编和反编译工具}
工具链包含了很多的工具,如果静态反汇编,可以使用\lstinline!objdump!;如果动态运行起来,在\lstinline!gdb!中也可以使用\lstinline!disass!反汇编。

对ARM反汇编支持最好的工具是IDA Pro,但它是商业软件。自由开源的反编译工具可以考虑radare\footnote{\href{http://www.radare.org/y/}{http://www.radare.org/y/}}或者smiasm\footnote{\href{http://code.google.com/p/smiasm/}{http://code.google.com/p/smiasm/}},但radare和smiasm对Thumb、Thumb-2指令集的支持都不强。

对这三个工具在ARM反汇编上的对比可以参考我的文章\footnote{\href{http://blog.claudxiao.net/2011/12/arm-disassemblers}{http://blog.claudxiao.net/2011/12/arm-disassemblers}}。

事实上,ARM的指令编码并不复杂,自己写一个ARM反汇编器的工作量远小于想象。

目前ARM的反编译工具只有Hex-Rays ARM Decompiler,是一款昂贵的收费软件。
\subsection{远程调试}
可以使用\lstinline!gdb!远程调试ARM设备或虚拟机中的程序。假设设备的IP地址是192.168.0.2,要调试的程序名为demo,则在设备端运行:
\begin{lstlisting}[language=bash, numbers=none]
  $ gdbserver 192.168.0.2:23456 /path/to/demo
\end{lstlisting}

接下来,在主机端使用ARM工具链中的gdb调试:
\begin{lstlisting}[language=bash, numbers=none]
  $ arm-eabi-gdb /path/to/demo
\end{lstlisting}

在进去gdb调试会话后,键入:
\begin{lstlisting}[language=bash, numbers=none]
  (gdb) target remote 192.168.0.2:23456
\end{lstlisting}

即可开始调试。

也可以使用IDA Pro远程调试Android系统,Berry写过一篇文章介绍操作方法\footnote{\href{http://debugman.com/thread/6230/1/1}{http://debugman.com/thread/6230/1/1}}。
\section{ARM体系结构}
\subsection{ARM指令集}
ARM同时是三个不同事物的名字:
\begin{itemize}
\item 一个公司
\item 一种体系结构
\item 一系列CPU产品
\end{itemize}
这一章主要是指第二种含义。

ARM是32位RISC结构,在ARMv5以后,基本采用Harvard结构,与传统的冯诺依曼结构不同的是,数据和代码被最大程度隔离了。既便于数据段保护的x86也有本质区别,ARM的代码和数据采用不同的总线传输(并因此获得并行而提速)。其数据段自然不能被执行。

\subsubsection{寄存器}
对程序员可见的寄存器主要是r0到r15共16个,不同的寄存器有不同的用途,将在下一节说明。

此外,有程序状态寄存器PSR,其中包括算数逻辑运算标志、执行状态位、当前中断号等。

\subsubsection{数据操作指令}
基本形式是:
\begin{lstlisting}[numbers=none]
<opcode>{<cond>}{S} <Rd>, <Rn>{, <operand2>}
\end{lstlisting}
其中:
\begin{itemize}
\item[opcode] 比如MOV、ADD,与x86类似,具体可以查ARM手册
\item[cond] 可选的条件码。条件执行是ARM的特色之一,几乎所有的ARM指令都可以加上条件码,实现条件执行。例如,EQ、LE等
\item[S] 可选的S位,如果该条指令会更改PSR,则应置上S位
\item[operand] 操作数可以是寄存器,也可以是一个右值(后面会说明),每条指令有多少操作数,分别是什么含义,应查手册
\end{itemize}

ARM中的右值操作数可以是一个寄存器、一个立即数,或者一个寄存器的移位。例如,右值R1, LSL \#2由R2的值逻辑左移2位得到,右值R1, ASR R3由R1的值算术右移R3位得到。

\subsubsection{分支跳转指令}
\begin{itemize}
\item[B] 直接跳转到一个地址,唯一的参数是12位立即数,与x86中的jmp功能一样
\item[BL] 将下一条指令地址(即返回地址)赋给LR寄存器,然后跳转到参数指定的地址,参数是一个12位立即数。BL与x86中的call功能一样
\item[BX] 参数是一个寄存器,跳转到该寄存器指向的地址
\end{itemize}
所有这些分支跳转指令都可以使用条件码,成为相应的条件跳转指令。

\subsubsection{内存访问指令1}
ARM使用简单的访存模型,所有内存读写都通过LDR和STR两条指令完成,而且只能在寄存器与内存之间读写,不能直接从内存读写到内存。LDR和STR都有一个可选的后缀B,不选时按字(4字节)读写,选择B时按字节读写。其第一个参数是寄存器,第二个参数是内存地址。

\subsubsection{寻址方式}
第一类寻址方式:
\begin{itemize}
\item 寄存器加上立即数偏移:[reg, \#$\pm$imm12]
\item 寄存器加上寄存器偏移:[reg, $\pm$reg]
\item 寄存器a加上移位后的寄存器b偏移:[rega, $\pm$regb, shift]
\end{itemize}
这些地址符号后面可以选择一个叹号:!。如果加上,表明先根据寻址规则修改寄存器,然后根据寄存器中的值访问内存;如果不加叹号,表示直接根据寻址规则访问内存。

第二类寻址方式则是先根据寄存器中的值访问内存,然后按照相应的规则更新寄存器:
\begin{itemize}
\item 访存后,寄存器加上立即数:[reg], \#$\pm$imm12
\item 访存后,寄存器加上寄存器:[reg], $\pm$reg
\item 访存后,寄存器a加上移位后的寄存器b:[rega], $\pm$regb, shift
\end{itemize}

\subsubsection{内存访问指令2}
还可以一次性读写多个字:
\begin{lstlisting}[numbers=none]
LDMcdum reg!, mreg
STMcdum reg!, mreg
\end{lstlisting}
其中:
\begin{itemize}
\item[cd] 可选的条件
\item[um] 访问模式,分别为IA读写后增加寄存器值、DA读写后减少寄存器值、IB读写前增加寄存器值、DB读写后增加寄存器值
\item[!] 表示会修改寄存器,修改方法参考um
\item[mreg] 支持一次指令多个寄存器,例如{R0-R3, R7, R10}
\end{itemize}

\subsection{ATPCS}
\subsubsection{名词解释}
例程(routine)、子例程(subroutine):对于一段可以调用、可以返回、保证栈平衡的代码片段,例程指调用者,子例程指被调用者。

过程(procedure):不返回结果值的例程。

函数(function):返回结果值的例程。

变量大小、内存对齐、字节序、复合类型等略。
\subsubsection{寄存器的用途}

ARM中有16个通用寄存器r0 – r15。其中:
\begin{itemize}
\item[r0 – r3] 用于传递参数、返回函数结果,因此又名a0 – a3。在例程内部也被用于保存临时结果
\item[r4 – r11] 用于保存例程内的局部变量值,又名v0 – v8。其中,r9(v6)是一个平台相关的值,不同的ARM平台必须为这个寄存器赋予特殊的含义
\item[r12 – r15] 有专门的用途,后面介绍。常用别名:IP、SP、LR、PC
\end{itemize}
\subsubsection{栈结构}

r13(SP)是栈指针。ARM中使用向下的满栈(full-descending),即SP始终指向最后一个已进入栈的数据,每次压栈时,SP自减需要的内存大小,然后将值存到SP新的位置。

也就是说,对栈的push相当于STMDB,对栈的pop相当于LDMIA,但一般不用这个后缀,而用STMFD和LDMFD。

SP的值应在栈的有效区间内,且mod 4 = 0。栈上有例程的调用帧结构。不要轻易地修改这个值。

\subsubsection{子例程调用}

ARM和Thumb指令集均有一个BL指令,它的操作是:将BL指令顺序下一条指令的地址(即返回地址)送入链接寄存器LR(r14),然后将调用的目的地址(即子例程地址)送入寄存器PC(r15)。

如果在Thumb中调用BL,则LR的第0位被设为1,否则被设为0。(因为指令地址要4字节对齐,因此LR的第0和1位都没有用。)

子例程返回很简单,将LR的值送入PC即可。

任何模拟上述过程的指令序列也会起到BL的效果,例如:mov LR, PC; BX r4。注意,任何时候读PC,得到的值是当前指令地址+8(why?请自己google)。

\subsubsection{结果返回}
\begin{itemize}
\item 不超过4字节的结果,一律用r0返回
\item 超过4字节的基本类型,继续用r1, r2, r3
\item 超过4字节的复合类型,或动态大小的结果,存储在内存中,并将其地址作为调用参数之一传入
\end{itemize}
\subsubsection{参数传递}

使用r0 – r3传递参数,如果不够,使用栈。

语言中的数据类型会按照相关标准转化为机器数据类型。
\section{ARM漏洞利用的特点}
与x86相比,ARM下要做到漏洞利用存在以下问题:
\begin{itemize}
\item 从v5开始,ARM普遍采用哈佛结构,数据段不可执行
\item 函数调用时,参数传递不再通过栈,而是使用寄存器
\item 返回地址按照ATPCS是通过LR寄存器传递,大部分被掉用函数会在其代码开始时将LR再次保存到栈上,在返回前从栈中直接取回至PC寄存器
\item 由于体系结构的本质区别,作为跳板的指令无法使用x86下常见的那些,而要根据实际需要来寻找
\end{itemize}
但也有好的消息。在Windows/x86平台引入DEP以后,因为代码部分的不可执行,出现的ret2libc和ROP等技巧,其思路也可以用于ARM。此外,ARM单条指令的描述能力是超过x86的,所以要寻找到合适的跳板指令序列并不是太难。

ARM漏洞利用是一个新的领域,目前参考文献较少,主要有\cite{arm_exploiting_linux, arm_stack_exploitation, arm_exploitation, arm_ropmap, arm_alphanumeric}。
\section{简单的示例}
\subsection{源代码}
下面我们看一个实际的例子。这个程序来自于\cite{arm_exploiting_linux},但因为编译源码使用的系统和编译器不同,后面的细节与原文有较大差异。
\lstinputlisting[language=c,caption={存在漏洞的示例代码}]{code/arm.test.c}

\subsection{反汇编结果}
如果使用的Toshiba AC100做实验,在Ubuntu下,从6.10开始gcc默认开启了-fstack-protector开关,存在不安全拷贝函数调用的函数,在退出前会做栈有效性检测(\_\_stack\_chk\_fail),因此进一步的漏洞利用会失败。应该在gcc编译时使用-fno-stack-protector来关闭这个开关。此外,在Toshiba AC100中,gcc默认将目标代码优化为Thumb指令集。后面的分析是基于ARM指令集的。可以给gcc加上-marm开关来强制指定其编译为ARM指令集。最后,也就是说,在AC100下,应该这样编译:

\begin{lstlisting}[language=bash, numbers=none]
  $ gcc test.c -o test -marm -fno-stack-protector
\end{lstlisting}

在编译完成后,就可以使用gdb来反汇编vuln()函数了。下面是在ARMEL的Debian系统下,用gcc4.4编译的结果:

\lstinputlisting[caption={vuln函数的反汇编结果}]{code/arm.test.dis1.asm}

逐一解析这些代码,以熟悉ARM指令。

在第3行,先后push了r11和lr寄存器。r11又称fp(帧指针寄存器),相当于x86下的ebp寄存器。它在该函数中使用了,所以将原来的值保存在栈中。lr前面有介绍,在调用该函数时将返回地址保存在了其中,因为在vuln()中还要bl到strcpy,还要用到lr,所以将lr保存在栈中,直到函数返回时(13行),才从栈中取出。这一点需要额外说明,虽然ATPCS规定由lr保存返回地址,且各编译器也遵循了这一约定,但在实现中,由于存在漏洞的函数其内部必然调用了其他函数,因此lr一定会在该函数的初始化时被保存到栈上,因此溢出是一定可以覆盖到它的。也就是说,虽然ATPCS的返回方法和x86的约定不同,但实际情况还是差不多。

第4行,给r11(fp)赋值。再次强调,ARM使用向下递减的满栈,因此r11指向sp加4,即栈顶第二个元素的地址。

第5行,sp自减24,开出了24字节的局部变量缓冲区。注意这里的\#24的立即数表示是十进制,而不是十六进制。

第6、7、10行,r0传给r3,r3传给r1。回忆ATPCS,r0中存着vuln()函数的第一个参数,即char *arg的值。调用strcpy时,r1是第二个参数,因此,将arg作为了strcpy的第二个参数。

第8、9行,在局部变量缓冲区取了12个字节的缓冲,给r0,作为strcpy的第一个参数。有人会奇怪,在源码中我们为buff申请了10个字节的缓冲,为什么这里成了12。这是因为ARM的内存访问都是4字节对齐的。

第11行,调用了strcpy函数。

第12行,恢复栈指针。

第13、14行,恢复r11原来的值,恢复lr的值,跳转至lr以返回。但很多编译器不这么实现,而是直接pop \{r11, pc\}。

可以看到,如果在strcpy时,输入的数据超过12个字节,就会逐步覆盖栈中保存的原来的r11(超过4字节)和lr(再超过4字节)。而lr会在后面取出作为函数返回地址。这样,就有可能通过覆盖栈来改变指令执行流程。

\subsection{栈结构}
我们再来看看当正常运行到上面第12行,即执行完strcpy后,栈的情况。我们用1234作为参数。

\lstinputlisting[caption={正常拷贝后的栈结构}]{code/arm.test.dis2.asm}

可以看到,参数1234已经出现在栈上(0xbedb9794处),而返回地址位于0xbedb97a4处,为0x0084ac。如果输入数据越界,则其17到20字节将正好覆盖这一返回地址。这个结论与我们前面对代码的分析是一致的。

\subsection{构造溢出}
从上面的栈结构,我们可以构造溢出数据了。如下所示:

\lstinputlisting[caption={构造溢出数据}]{code/arm.test.dis3.asm}

我们先看下donuts()函数的地址,为0x00008438,再考虑不破坏其他栈结构,因此构造了长为20字节的输入数据。其前12个字节随便填充,13-16字节为原来栈上的r11值,17-20字节为我们想要的返回地址,即donuts()函数的地址。最后,可以看到donuts()函数确实运行了起来,程序也正常退出了。

\section{漏洞攻击方法}
上述示例只是一个演示性的栈溢出。如果要发起真实攻击,跳到一个本地函数并不是我们需要的。

在x86中,最初的做法是将shellcode作为数据传给程序,程序将其放在溢出后的栈上。当函数退出时,跳转到栈上执行shellcode。在引入DEP后,栈上数据不可执行,从而产生了包括ret2libc在内的ROP技术,即跳转到一些系统API或库函数,通过控制这些API的参数实现一定的能力。

在x86上,参数是通过栈传递的。也就是说,通过栈溢出几乎一定可以把参数安置好,并调用其他函数。但在ARM上这一招没法直接用了。ATPCS规定使用r0 - r3这四个寄存器传递参数,而通常栈溢出是无法影响到这些寄存器的。只能寄希望于在溢出后,存在这类指令序列:将栈上的值拷贝至r0 - r3中。

以此思想为基础,Avraham提出了所谓的Ret2ZP(Return to Zero Protection)的攻击方法,主要在\cite{arm_stack_exploitation}和\cite{arm_exploitation}中进行了阐述。

\section{zergRush分析}
\subsection{背景和原理}
Revolutionary工具开发小组在2011年10月发布了一个在Android 2.2和2.3上获得root权限的方法\footnote{\href{http://forum.xda-developers.com/showthread.php?t=1296916}{http://forum.xda-developers.com/showthread.php?t=1296916}},并公布了漏洞利用代码zergRush.c\footnote{\href{https://github.com/revolutionary/zergRush/blob/master/zergRush.c}{https://github.com/revolutionary/zergRush/blob/master/zergRush.c}}。tomken\_zhang已经在其博客上发表了两篇文章对其分析\footnote{\href{http://blog.csdn.net/tomken\_zhang/article/details/6866260}{http://blog.csdn.net/tomken\_zhang/article/details/6866260}\newline\href{http://blog.csdn.net/tomken\_zhang/article/details/6870104}{http://blog.csdn.net/tomken\_zhang/article/details/6870104}}。本文做进一步梳理和补充。

产生漏洞的主要原因是:具有root权限的vold进程使用了libsysutils.so库,该库的一个函数存在栈溢出,因此可以在root权限执行输入的shellcode。

存在漏洞的函数为FrameworkListener::dispatchCommand,位于源码的$$\backslash system\backslash core\backslash libsysutils\backslash src\backslash FrameworkListener.cpp$$中,其中的局部变量argv为固定大小的指针数组,当输入参数的数量超过其大小时,会越界写入栈中。

zergRush.c成功地利用了这一漏洞,并进一步:
\begin{enumerate}
\item 在/data/local/tmp/下增加一个置了S位的shell;
\item 使Android中后续启动的adb进程以root权限运行。
\end{enumerate}

其中第二步的方法是:adb进程最初以root运行,之后调用setuid()降低权限\footnote{\href{http://blog.claudxiao.net/2011/04/android-adb-setuid/}{http://blog.claudxiao.net/2011/04/android-adb-setuid/}}降权之前,会判断系统属性ro.kernel.qemu,如果该属性位1,则不降权。

\subsection{函数功能概要}

\begin{description}
\item[die] 打印出错信息,退出程序
\item[copy] 将一个文件拷贝为另一个文件
\item[remount\_data] 重新mount一个分区
\item[find\_symbol] 查找libc.so中导出函数的内存地址
\item[check\_addr] 确定一个地址中是否包含被禁止的字节
\item[do\_fault] 构造溢出数据和exploit,并通过socket发送给vold进程
\item[find\_rop\_gadgets] 从libc.so中寻找两个特殊指令序列的地址
\item[checkcrash] 调用do\_fault,判断其溢出产生的调试信息中是否包含sp
\item[find\_stack\_addr] 调用do\_fault,从其溢出产生的调试信息中定位栈地址
\item[do\_root] 将shell文件的S位置上,并设置ro.kernel.qemu属性为1
\item[main] 主函数,完成漏洞利用的所有步骤
\end{description}

\subsection{main函数}
\lstinputlisting[language=c,caption={zergRush的main函数}, firstnumber=387]{code/zergRush.main.c}
\begin{itemize}
\item[395-396] 如果当前程序是以root权限运行,并且程序名为boomsh,则调用do\_root,执行附加的两步操作
\item[402-405] 将自身拷贝至/data/local/tmp/boomsh,并设置其权限为0711,将/system/bin/sh拷贝至/data/local/tmp/sh。
\item[407-408] 根据/system/bin/vold文件的大小获得其对应进程中堆的大概地址heap\_addr。
\item[410-421] 根据系统版本对heap\_addr做微调。如果不是2.2或2.3系统,退出。
\item[423-428] 查询libc.so中system调用的地址,保存至system\_ptr。
\item[430-443] 通过checkcrash函数,判断buffsz为16或24时能否成功利用。这里buffsz实际指libsysutils中造成栈溢出的指针数组argv的容量。
\item[445-484] 调用find\_stack\_addr函数,确定栈地址。反复尝试五次,每次对堆地址heap\_addr做微调,直至成功。判断得到的栈地址是否有效。
\item[486-487] kill掉当前的logcat进程,删除/data/local/tmp/crashlog文件。
\item[489-491] 调用find\_rop\_gadgets函数,在libc.so中寻找指令序列add sp, \#108; pop \{r4-r7, pc\},将地址保存在stack\_pivot;寻找指令pop \{r0, pc\},将地址保存在pop\_r0。
\item[493-514] 尝试三次,每次调用do\_fault,之后判断/data/local/tmp/sh的S位是否置上,一旦置上,则利用成功;否则,微调栈地址heap\_addr(加减16)。
\item[516-533] 一旦利用成功,并且系统的ro.kernel.qemu属性已经被置为1,则利用完成,重启的adb进程即可获得root权限。
\end{itemize}

\subsection{do\_root函数}
\lstinputlisting[language=c,caption={zergRush的do\_root函数}, firstnumber=377]{code/zergRush.doroot.c}
\begin{itemize}
\item[395-396] 若当前程序是以root权限执行的/data/local/tmp/boomsh,则调用do\_root函数。
\item[379] 重新mount目录/data。
\item[380] 将/data/local/tmp/sh的所有者设置为root。
\item[381] 将/data/local/tmp/sh的属性设置为04711,注意其S位被置位。
\item[382] 设置系统的ro.kernel.qemu属性为1。
\end{itemize}

\subsection{find\_stack\_addr函数}
\lstinputlisting[language=c,caption={zergRush的find\_stack\_addr函数}, firstnumber=324]{code/zergRush.findstackaddr.c}
\begin{itemize}
\item[332-333] 清空logcat缓存,删除老的/data/local/tmp/crashlog日志文件。
\item[335-340] 重启一个logcat,将其日志输出至/data/local/tmp/crashlog文件。
\item[342-349] 调用一次do\_fault,等待3秒后,读取crashlog文件中的logcat日志。
\item[350-366] 搜索logcat日志中的debug信息,"4752455a"之前8个字节为栈基址stack\_addr,"5245564f"往之前8个字节为over,"sp"之后5个字节为栈顶sp。
\item[370-371] jumpsz = over – sp
\end{itemize}

\subsection{do\_fault函数}
\lstinputlisting[language=c,caption={zergRush的do\_fault函数}, firstnumber=163]{code/zergRush.dofault.c}
\begin{itemize}
\item[165] buf是最后发送至vold的shellcode。
\item[169-181] padding是shellcode中的一段填充内容,全部为Z,无意义。长度为padding\_sz + 12。padding\_sz由108减去jumpsz计算得到。
\item[183-184] 通过socket连到本地的vold进程。
\item[186-190] 将栈地址stack\_addr、指令序列1地址stack\_pivot、指令序列2地址pop\_r0、system调用地址system\_ptr、堆地址heap\_addr,分别填充到相应的字符串中。
\item[192-198] 开始构造shellcode。注意第195行,这里根据buffsz,也就是尝试出来的溢出数组argv的大小,构造相应数量的输入参数。
\item[200-201] 计算一下/data/local/tmp/boomsh字符串将会出现的地址,这个字符串会作为shellcode的一部分发到栈中,因此可以根据栈地址和偏移计算出来,最后作为system调用的参数。
\item[208] 把上述地址转为字符串s\_bsh\_addr。
\item[209] 进一步构造shellcode,包括栈地址、堆地址、填充、指令序列2地址、boomsh字符串地址、system调用地址、boomsh字符串等。
\item[214] 将shellcode发送至vold进程。
\end{itemize}

\subsection{总结}

综合来看,zergRush.c的思路如下:
\begin{enumerate}
\item 计算出vold的堆地址
\item 查到system调用的地址
\item 尝试出栈缓冲区大小
\item 通过崩溃产生的调试信息,取得栈地址和栈结构信息
\item 在libc.so中找寻跳板指令
\item 根据缓冲区大小、栈结构和上述各种地址,构造出有效的shellcode来,发送到vold
\item shellcode在vold中以root权限运行,它通过system调用运行该利用程序的一个副本boomsh
\item 程序副本boomsh以root权限运行时,会置上shell程序的S位,并设置系统属性ro.kernel.qemu
\item 结束掉adb,后续开启的adb进程将具有root权限
\end{enumerate}

非常典型的缓冲区溢出利用思路,但与PC相比,利用了android中几个特殊之处:
\begin{itemize}
\item vold的溢出会在adb logcat中输出调试信息,这些信息说明了其内存结构,而其他程序可以读取到这些信息;
\item 在ARM架构下,跳板指令有了更多的选择,ret2libc的攻击也可能更容易实现
\item adb的降低权限过程又一次被利用。
\end{itemize}

最后,我们没有进一步分析shellcode的详细结构和跳转过程,难度已经不大。反而是libsysutils.so这个通用库中的溢出有没有可能造成其他问题,需要进一步分析。


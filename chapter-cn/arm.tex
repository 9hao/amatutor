%!TEX encoding = UTF-8 Unicode
\chapter{ARM漏洞利用}
\section{学习的目的}
目前绝大部分的智能移动终端使用ARM处理器。包括使用Android、Apple iOS、Symbian、Windows Mobile/Windows Phone、RIM OS等操作系统的手机和平板电脑。

此外,ARM还被用于大量的嵌入式设备,在PC所不能及之处发挥着重要的作用。

PC也可以使用ARM。几年前,Debian就发布了ARM版,2011年Ubuntu也开始支持ARM。Windows 8也将被微软移植到ARM上。

ARM还将向服务器市场发展,目前还出现了基于ARM和GPU的超级计算机。

可以看出,ARM设备无论是数量还是覆盖面都不逊色于PC,但ARM的安全问题,尤其是其上软件漏洞的问题,直到最近几年才开始引起人们的注意。
\section{环境和工具}
\subsection{容易获得的ARM系统}
如果需要自己编写的ARM代码实际运行起来,或者调试这些代码,就需要一个能运行ARM代码的系统。ARM的开发板并不贵,但还有更简单的方法获得这样一个系统。
\subsubsection{Android模拟器或手机}
Android的底层是Linux内核,它运行于ARM之上。使用Android学习ARM漏洞利用的优点是:
\begin{itemize}
\item 如果就是要做Android安全,这自然是最佳的选择
\item 有完善的开发工具和系统源码
\item Android模拟器的硬件(边际)成本为0
\end{itemize}
但缺点也不少:
\begin{itemize}
\item 缺乏丰富的本地软件
\item 不使用glibc库,而是自行开发的bionic,因此一些利用技巧在细节上会和Linux(on ARM)有差异
\item 目标文件的编译,或者完全静态链接,或者用NDK开发,或者与Android源码一起编译,均显繁琐
\end{itemize}

无论如何,这种零成本的环境,还是强烈推荐常备着。
\subsubsection{Debian虚拟机}
Debian很早就开始了对ARM处理器的移植,目前几乎所有的软件都有了ARM版本,因此Linux用户常用的工具(例如objdump、gdb等)都可以直接在本地运行。

另一方面,qemu模拟器已经可以完整地模拟ARM处理器及其硬件平台,可以用它在x86的PC上运行ARM版的Debian。

可以直接从Debian的官网下载到其ARM安装包,但建议使用下列方案:
\hrefurl{http://www.aurel32.net/info/debian\_arm\_qemu.php}

更省事的方案是直接下载已经安装好的qemu镜像,地址是:
\hrefurl{http://people.debian.org/~aurel32/qemu/arm/}
\subsubsection{Toshiba AC100上网本}
Toshiba AC100是目前唯一一款拥有标准全键盘的ARM笔记本。它最初是为Android设计,但爱好者已经将多个Ubuntu移植到该机器上。最终,Ubuntu官方从11.10版本开始专门为这一机器提供预编译版本和软件源。

该机器在国内的型号为AC100-01B,已经不在出新货,可以从爱好者手中购买到二手的,2011年年底的行情为750元左右。

这个笔记本是目前唯一一种可以最轻松获得的真实ARM机器。拥有前面Debian虚拟机的一切优点。
\subsubsection{Nokia N900手机}
另一个hack利器是Nokia N900手机,这款老手机配置较高,采用已经被抛弃的Maemo系统,该系统是基于Linux的手机操作系统,开放性较高。N900的键盘操作性不强,系统也适合于有hack精神的人玩。
\subsection{交叉编译工具}
进行嵌入式开发时,编译工具运行于一个平台,但生成另一个平台的指令,这种编译过程称之为交叉编译,所使用的编译工具又称之为工具链。

在x86上编译ARM代码最常用的工具链由CodeSourcery公司免费提供,有Windows版本和Linux版本,其下载地址是:\hrefurl{http://www.mentor.com/embedded-software/sourcery-tools/sourcery-codebench/editions/lite-edition}

此外,在基于Debian的系统(例如Ubuntu)中,也可以通过apt-get install gcc-arm-linux-gnueabi直接获得一个交叉编译gcc。

Android的NDK中实际上也包含了完整的工具链,包括编译器、调试器、binutils等。它有Windows和Linux版本,下载地址是:
\hrefurl{http://developer.android.com/sdk/ndk/index.html}
然而NDK也有不足,即只能为Android编译代码,也只能使用其提供的有限的库接口。对后面一个问题,如果需要编译使用了Andrid系统中较底层API的ARM程序,建议使用agcc工具,具体可以看我写的文章:
\hrefurl{http://blog.claudxiao.net/2011/10/android\_agcc/}
\subsection{反汇编和反编译工具}
工具链包含了很多的工具,如果静态反汇编,可以使用objdump;如果动态运行起来,在gdb中也可以使用disasm反汇编。

最ARM反汇编支持最好的工具是IDA Pro,但它是商业软件,自由开源的反编译工具可以考虑radare:
\hrefurl{http://www.radare.org/y/}
或者smiasm:
\hrefurl{http://code.google.com/p/smiasm/}
但radare和smiasm对Thumb、Thumb-2指令集的支持都不强。

事实上,ARM的指令编码并不复杂,自己写一个ARM反汇编器的工作量远小于想象。

目前ARM的反编译工具只有随IDA Pro 6.2一起发布的Hex-Rays ARM Decompiler,是一款昂贵的收费软件。
\subsection{远程调试}
gdb远程调试

IDA远程调试请参考:
\hrefurl{http://bbs.pediy.com/showthread.php?t=141739}

\section{ARM体系结构}
\section{ARM漏洞利用的特点}
\section{一个示例}
\section{Ret2ZP攻击}
\section{setuid漏洞分析}
\section{zergRush漏洞分析}
%!TEX encoding = UTF-8 Unicode
\chapter{前言}
\CTEXsetup[format={\large\bfseries}]{section}
\section*{目的}
Android已经成为恶意代码的肆虐之地,安全工程师与恶意攻击者的艰苦对抗将在这一平台延续,双方人数上的显著差异也会长期存在。这一切正如在PC上曾经发生过的那样。

更为不利的是,恶意代码作者之间并不存在直接的利益竞争关系,而反病毒厂商却在商界拼得你死我活。这导致攻击技术会在攻击者之间广泛交流,而防御技术却始终是各企业的核心技术而被保密。

为了在天平的正义一侧增加砝码,笔者认为应该将一些技术梳理并公开,推动安全工程师在技术上的交流合作与共同进步。笔者是这一领域的初学者,将学习过程写成笔记,最后逐渐汇聚成为这本小册子,只希望能有抛砖引玉之效。

诚然,反病毒领域绝大部分技术都是双刃剑。很多知识的公开和传播会被更多潜在的攻击者参考和学习。但笔者相信,在目前的局面下,公开交流所带来的边界收益会比它可能带来的损失要大得多。

这本小册子中的技术既可以被安全工作者用于反病毒和软件保护,也可能被恶意攻击者用于病毒对抗和软件破解。如何使用,在于人心。我们应该追随自己内心的道德准则,用它们造福于头顶这片星空下的人们。

swordlea说过一句话:“不需要懂得信息安全的人是幸福的,而我们的职责是保障他们的幸福。”谨以此与各位共勉。

\section*{致谢}
通常在事情完成后才总结致谢,但这本小册子会在何时完成现在很难判断,而且这个新兴领域的持续发展可能促使它不断更新下去,因此即便目前还在草稿阶段,也需要先写致谢。

从2011年提笔至今,有很多朋友对我在这一领域的学习、工作和写作提供过各种帮助,或者产生影响,他们是:tompanpan, sworldlea, DemonHunter, berry, hip。在此向他们的指导和帮助表示衷心的感谢。

不少前辈在前言中说:“任何一本书都不是一个人能完成的。”我深有感触。虽然在这本小册子的封面大胆地写上自己姓名,但如果没有这些朋友,它绝对无法成为现在的模样。

我还希望向其中涉及的各种工具、系统、设备的开发者们表示致敬。正是整个社区的共同努力,让安全工程师拥有了各种利器、能够一展所长。

\section*{反馈}
这本小册子中肯定存在很多错误和不足之处。如果您有任何建议、看法,希望能不吝指教。我的邮箱是:\href{mailto:xiaozihang@gmail.com}{xiaozihang@gmail.com},期待您的来信,并先表示万分的感谢。

部分章节的草稿会在博客\url{http://blog.claudxiao.net}上先发布,欢迎订阅和留言。

\section*{版权}
这本小册子的{\LaTeX}源文件公开发布在\url{http://code.google.com/p/amatutor}。这些源文件本身(包括其中的\lstinline!.tex!文件、代码片段、图片等)以及由它们编译产生的PDF文件中的实际内容,除申明引用自别处外,均系作者原创。

这些文件或内容均遵循“署名-非商业性使用-禁止演绎 2.5 中国大陆 (CC BY-NC-ND 2.5)”协议发布。也就是说,您可以自由地复制、发行、展览、表演、放映、广播或通过信息网络传播本作品,惟须遵守下列条件:

\begin{itemize}
  \item 署名 — 您必须按照作者或者许可人指定的方式对作品进行署名
  \item 非商业性使用 — 您不得将本作品用于商业目的
  \item 禁止演绎 — 您不得修改、转换或者以本作品为基础进行创作
\end{itemize}

该协议的完整法律文本可以查阅\url{http://creativecommons.org/licenses/by-nc-nd/2.5/cn/legalcode}。

根据上述协议的框架,对本作品中全部或部分内容的出版或引用,包括网络和纸质两种传播方式,如果是盈利性目的,或实际直接产生盈利效应,须经作者书面同意。

\CTEXsetup[format={\Large\bfseries\centering}]{section}
